\section{Ejercicio 12.16}
\subsection{Modelo}
En esta extensi'on del ejercicio anterior debemos considerar los nuevos generadores, la reserva de agua y como ambas cosas afectar'an al objetivo y las restricciones de demanda anteriores. 
\subsubsection{Variables}
Agregamos las siguientes variables para cada periodo P:
\begin{itemize}
\item $Reservoir\_PeriodP$, representando la medida de la reserva en el periodo P
\item $Pumping\_PeriodP$, representando los metros que se modifica la reserva en el periodo P
\item $RunningHA\_PeriodP$ y $RunningHB\_PeriodP$, representando si los generadores hidroel'ectricos A y B estan operando en el periodo P (variables booleanas)
\item $StartHA\_PeriodP$ y $StartHB\_PeriodP$, representando si los generadores hidroel'ectricos A y B son encendidos en el periodo P (variables booleanas)
\end{itemize}
\subsubsection{Restricciones}
\subsubsection{Objetivo}
\subsection{Soluci'on}