\section{Ejercicios Pr'actica 8}
\subsection{Ejercicio 20.a}
\subsubsection{Enunciado}
Encontrar una desigualdad v'alida para $X$ que corte el punto $x^{*}$ dado, donde
\begin{center}
$X = \{x \in B^{5} / 9x_1 + 8x_2 + 6x_3 + 6x_4 + 5x_5 \leq 14\}$ \\
$x^{*} = (0, 5/8, 3/4, 3/4, 0)$
\end{center}
\subsubsection{Soluci'on}
Utilizando el m'etodo de Chv'atal-Gomory podemos multiplicar la desigualdad por $1/6$ y luego aplicar la funci'on $\left \lfloor{}\right \rfloor$ a cada coeficiente:
\begin{center}
\begin{equation}
9/6 x_1 + 8/6 x_2 + 6/6 x_3 + 6/6 x_4 + 5/6 x_5 \leq 14/6
\end{equation}
\begin{equation}
3/2 x_1 + 4/3 x_2 + x_3 + x_4 + 5/6 x_5 \leq 7/3
\end{equation}
\begin{equation} \label{eq:solve}
x_1 + x_2 + x_3 + x_4 \leq 2
\end{equation}
\end{center}
Si ahora evaluamos la desigualdad ~\ref{eq:solve} en $x^{*}$ tenemos:
\begin{center}
$0 + 5/8 + 3/4 + 3/4 = 17/8 > 2$
\end{center}
Entonces ~\ref{eq:solve} es una desigualdad v'alida y corta a $x^{*}$.

\subsection{Ejercicio 22}
\subsubsection{Enunciado}
Dado un grafo $G = (V, E)$ con $n = |V|$ considerar el conjunto de vectores soluciones de $X$ del
problema de conjunto independiente $X = \{ x\in B^{n}/ x_{i}+x_{j}\leq 1$ para toda arista $e =(i,j) \}$. Mostrar que la desigualdad de clique $\sum_{j\in C} x_{j} \leq 1$ para $C$ clique maximal de $G$, es v'alida. 
\subsubsection{Soluci'on}
Lo demostraremos por el absurdo. Si la desigualdad de clique fuese inv'alida, entonces tendriamos:
\begin{equation}
\sum_{j\in C} x_{j} > 1
\end{equation}
 Con lo cual existen $r, s \in C$  tales que:
\begin{equation}
x_{r} + x_{s} = 2 > 1
\end{equation}
Pero como son parte de C clique maximal de G, $r$ y $s$ son adyacentes. Es decir, existe una arista $e' = (r,s)$, pero esto es absurdo ya que entonces no pertenecer'ian al conjunto independiente.
Por la tanto la desigualdad es v'alida.