\section{Ejercicio 12.13}
\subsection{Modelo}
\subsubsection{Variables}
Las variables ${D1}_i$, i = 1, .. ,23 representan si el retailer i pertenece al grupo ${D}_1$. Son variablres enteras.
${D1}_1$
${D1}_2$
${D1}_3$
${D1}_4$
${D1}_5$
${D1}_6$
${D1}_7$
${D1}_8$
${D1}_9$
${D1}_{10}$
${D1}_{11}$
${D1}_{12}$
${D1}_{13}$
${D1}_{14}$
${D1}_{15}$
${D1}_{16}$
${D1}_{17}$
${D1}_{18}$
${D1}_{19}$
${D1}_{20}$
${D1}_{21}$
${D1}_{22}$
${D1}_{23}$  \\

Variables slack para las restricciones. Las variables plusXXX y minusXXX son las variables slack que me permiten que los valores
controlados por el grupo $D_1$ pueda variar entre 35 y 45 porciento del total a controlar en cada area. \\
Variables slack sobre el control de Delivery Points: \\
$minusDelivery$ \\
$plusDelivery$ \\
Variables slack sobre el control de los Oil Markets en la region 1: \\
$minusOil1$ \\
$plusOil1$ \\
Variables slack sobre el control de los Oil Markets en la region 2: \\
$minusOil2$ \\
$plusOil2$ \\
Variables slack sobre el control de los Oil Markets en la region 3: \\
$minusOil3$ \\
$plusOil3$ \\
Variables slack sobre el control del Spirit Market: \\
$minusSpirit$ \\
$plusSpirit$ \\
\subsubsection{Restricciones}
A continuacion las restricciones impuestas en el modelo y una breve explicacion sobre cada una. \\
El retailer i-esimo o pertenece o no pertenece a $D_1$. Solo puede tomar valores 0 o 1. \\
$0 \leq {D1}_1 \leq 1$ \\
$0 \leq {D1}_2 \leq 1$ \\
$0 \leq {D1}_3 \leq 1$ \\
$0 \leq {D1}_4 \leq 1$ \\
$0 \leq {D1}_5 \leq 1$ \\
$0 \leq {D1}_6 \leq 1$ \\
$0 \leq {D1}_7 \leq 1$ \\
$0 \leq {D1}_8 \leq 1$ \\
$0 \leq {D1}_9 \leq 1$ \\
$0 \leq {D1}_{10} \leq 1$ \\
$0 \leq {D1}_{11} \leq 1$ \\
$0 \leq {D1}_{12} \leq 1$ \\
$0 \leq {D1}_{13} \leq 1$ \\
$0 \leq {D1}_{14} \leq 1$ \\
$0 \leq {D1}_{15} \leq 1$ \\
$0 \leq {D1}_{16} \leq 1$ \\
$0 \leq {D1}_{17} \leq 1$ \\
$0 \leq {D1}_{18} \leq 1$ \\
$0 \leq {D1}_{19} \leq 1$ \\
$0 \leq {D1}_{20} \leq 1$ \\
$0 \leq {D1}_{21} \leq 1$ \\
$0 \leq {D1}_{22} \leq 1$ \\
$0 \leq {D1}_{23} \leq 1$ \\

El grupo $D_1$ tiene entre el 35 y el 45 porciento del total del total de Delivery Points \\

11 ${D1}_1$
+ 47 ${D1}_2$
+ 44 ${D1}_3$
+ 25 ${D1}_4$
+ 10 ${D1}_5$
+ 26${D1}_6$
+ 26 ${D1}_7$
+ 54 ${D1}_8$
+ 18 ${D1}_9$
+ 51 ${D1}_{10}$
+ 20 ${D1}_{11}$
+ 105 ${D1}_{12}$
+ 7 ${D1}_{13}$
+ 16 ${D1}_{14}$
+ 34 ${D1}_{15}$
+ 100 ${D1}_{16}$
+ 50 ${D1}_{17}$
+ 21 ${D1}_{18}$
+ 11 ${D1}_{19}$
+ 19 ${D1}_{20}$
+ 14 ${D1}_{21}$
+ 10 ${D1}_{22}$
+ 11 ${D1}_{23}$
+ $plusDelivery$ - $minusDelivery = $ XXXHACERLACUENTAXXX \\

El grupo $D_1$ suma entre el 35 y el 45 porciento del total del total de Spirit Markets \\

34 ${D1}_1$
+ 411 ${D1}_2$
+ 82 ${D1}_3$
+ 157 ${D1}_4$
+ 5 ${D1}_5$
+ 183${D1}_6$
+ 14 ${D1}_7$
+ 215 ${D1}_8$
+ 102 ${D1}_9$
+ 21 ${D1}_{10}$
+ 54 ${D1}_{11}$
+ 0 ${D1}_{12}$
+ 6 ${D1}_{13}$
+ 96 ${D1}_{14}$
+ 118 ${D1}_{15}$
+ 112 ${D1}_{16}$
+ 535 ${D1}_{17}$
+ 8 ${D1}_{18}$
+ 53 ${D1}_{19}$
+ 28 ${D1}_{20}$
+ 69 ${D1}_{21}$
+ 65 ${D1}_{22}$
+ 27 ${D1}_{23}$
+ $plusSpirit$ - $minusSpirit = $ XXXHACERLACUENTAXXX \\

El grupo $D_1$ suma entre el 35 y el 45 porciento del total del total de los Oil Markets en las regiones 1, 2 y 3 \\


9 ${D1}_1$
+ 13 ${D1}_2$
+ 14 ${D1}_3$
+ 17 ${D1}_4$
+ 18 ${D1}_5$
+ 19${D1}_6$
+ 23 ${D1}_7$
+ 21 ${D1}_8$
+ $plusOil1$ - $minusOil1 = $ XXXHACERLACUENTAXXX \\



+ 9 ${D1}_9$
+ 11 ${D1}_{10}$
+ 17 ${D1}_{11}$
+ 18 ${D1}_{12}$
+ 18 ${D1}_{13}$
+ 17 ${D1}_{14}$
+ 22 ${D1}_{15}$
+ 24 ${D1}_{16}$
+ 36 ${D1}_{17}$
+ 43 ${D1}_{18}$
+ $plusOil2$ - $minusOil2 = $ XXXHACERLACUENTAXXX \\


+ 6 ${D1}_{19}$
+ 15 ${D1}_{20}$
+ 15 ${D1}_{21}$
+ 25 ${D1}_{22}$
+ 39 ${D1}_{23}$
+ $plusOil3$ - $minusOil3 = $ XXXHACERLACUENTAXXX \\

El grupo $D_1$ tiene entre el 35 y el 45 porciento del grupo A, que por ser entero son exactamente 3 retailers \\


${D1}_1 + {D1}_2 + {D1}_3 + {D1}_5 + {D1}_6 + {D1}_{10} + {D1}_{15} + {D1}_{20} = 3$ \\


El grupo $D_1$ tiene entre el 35 y el 45 porciento del grupo B, que por tener que ser entero son exactamente 6 retailers \\


${D1}_4 + {D1}_7 + {D1}_8 + {D1}_9 + {D1}_{11} + {D1}_{12} + {D1}_{13} + {D1}_{14} + {D1}_{15} + {D1}_{16}
  + {D1}_{17} + {D1}_{18} + {D1}_{19} + {D1}_{21} + {D1}_{22} + {D1}_{23}   = 6$ \\


Por ultimo, las variables slack toman valores entre 0 y el 5 porciento del total de cada grupo.\\


$0 \leq minusDelivery \leq$ XXXHACERLACUENTAXXX \\
$0 \leq plusDelivery \leq$ XXXHACERLACUENTA \\
$0 \leq minusOil1 \leq$ XXXHACERLACUENTA \\
$0 \leq plusOil1 \leq$ XXXHACERLACUENTA \\
$0 \leq minusOil2 \leq$ XXXHACERLACUENTA \\
$0 \leq plusOil2 \leq$ XXXHACERLACUENTA \\
$0 \leq minusOil3 \leq$ XXXHACERLACUENTA \\
$0 \leq plusOil3 \leq$ XXXHACERLACUENTA \\
$0 \leq minusSpirit \leq$ XXXHACERLACUENTA \\
$0 \leq plusSpirit \leq$ XXXHACERLACUENTA \\
 \subsubsection{Funci\'on objetivo}
Lo primero que se pedia era verificar si existia una solucion factible. De ser posible, se pedian dos posibles objetivos.\\
El primer objetivo: Minimizar la suma de los desvios porcentuales de las separaciones 40/60.\\

Minimizar \\
$\frac{100}{LALALA} minusDelivery$ +
$\frac{100}{LALALA} plusDelivery$ +
$\frac{100}{LALALA} minusOil1$ +
$\frac{100}{LALALA} plusOil1$ +
$\frac{100}{LALALA} minusOil2$ +
$\frac{100}{LALALA} plusOil2$ +
$\frac{100}{LALALA} minusOil3$ +
$\frac{100}{LALALA} plusOil3$ +
$\frac{100}{LALALA} minusSpirit$ +
$\frac{100}{LALALA} plusSpirit$ \\

El segundo objetivo que se pedia era minimizar el maximo desvio porcentual.
Para esto se agrego al modelo una variable no entera: \\
$maxDesvio$ \\

Tambien se agregaron las siguientes restricciones: \\
$maxDesvios \leq \frac{100}{LALALA} minusDelivery$ +
$\frac{100}{LALALA} plusDelivery$ \\


$maxDesvios \geq \frac{100}{LALALA} minusOil1$ +
$\frac{100}{LALALA} plusOil1$ \\


$maxDesvios \geq  \frac{100}{LALALA} minusOil2$ +
$\frac{100}{LALALA} plusOil2$\\


$maxDesvios \geq \frac{100}{LALALA} minusOil3$ +
$\frac{100}{LALALA} plusOil3$ \\


$maxDesvios \geq \frac{100}{LALALA} minusSpirit$ +
$\frac{100}{LALALA} plusSpirit$ \\

 Por ultimo, la funcion objetivo se escribio de la siguiente forma: \\

 Minimizar\\
 $maxDesvios$

\newpage
\subsection{Resultados}

A continuacion el resultado devuelto por nuestra implementacion del modelo en cplex para el modelo junto con el primer objetivo.\\


Status = Optimal \\
Objective = 2.7976179692883845 \\
Variables: \\
D1\_1 = 1.0 \\
D1\_10 = 0.0 \\
D1\_11 = 1.0 \\
D1\_12 = 0.0 \\
D1\_13 = 0.0 \\
D1\_14 = -0.0 \\
D1\_15 = 0.0 \\
D1\_16 = 1.0 \\
D1\_17 = -0.0 \\
D1\_18 = 1.0 \\
D1\_19 = -0.0 \\
D1\_2 = 1.0 \\
D1\_20 = 0.0 \\
D1\_21 = 1.0 \\
D1\_22 = 1.0 \\
D1\_23 = 0.0 \\
D1\_3 = 1.0 \\
D1\_4 = 1.0 \\
D1\_5 = 0.0 \\
D1\_6 = -0.0 \\
D1\_7 = 0.0 \\
D1\_8 = 0.0 \\
D1\_9 = -0.0 \\
minusDelivery = -0.0 \\
minusOil1 = 0.0 \\
minusOil2 = -0.0 \\
minusOil3 = -0.0 \\
minusSpirit = 34.0 \\
plusDelivery = 0.0 \\
plusOil1 = 0.6000000000000014 \\
plusOil2 = 2.0 \\
plusOil3 = -0.0 \\
plusSpirit = -0.0 \\


Podemos ver que la solucion no solo es factible, sino ademas optima.\\
De esta forma, los retailers que conforman el grupo $D_{1}$ son los retailers con numero: 1,2,3,4,11,16,18,21,22\\

Al correr el modelo con el segundo objetivo se obtuvo el siguiente resultado: \\

Status = Optimal \\
Objective = 0.547945205479452 \\
Variables: \\
D1\_1 = 0.0 \\
D1\_10 = 1.0 \\
D1\_11 = 0.0 \\
D1\_12 = -0.0 \\
D1\_13 = 0.0 \\
D1\_14 = -0.0 \\
D1\_15 = 0.0 \\
D1\_16 = 1.0 \\
D1\_17 = -0.0 \\
D1\_18 = 1.0 \\
D1\_19 = 0.0 \\
D1\_2 = 1.0 \\
D1\_20 = -0.0 \\
D1\_21 = 1.0 \\
D1\_22 = 1.0 \\
D1\_23 = 0.0 \\
D1\_3 = -0.0 \\
D1\_4 = 1.0 \\
D1\_5 = 1.0 \\
D1\_6 = 0.0 \\
D1\_7 = -0.0 \\
D1\_8 = -0.0 \\
D1\_9 = 1.0 \\
maxDesvio = 0.547945205479452 \\
minusDelivery = 4.0 \\
minusOil1 = 0.7342465753424656 \\
minusOil2 = 1.0890410958904109 \\
minusOil3 = 0.0 \\
minusSpirit = 0.0 \\
plusDelivery = 0.0 \\
plusOil1 = 6.334246575342467 \\
plusOil2 = 0.08904109589041087 \\
plusOil3 = 0.0 \\
plusSpirit = 8.0 \\

Nuevamente podemos observar que la solucion no solo es factible, sino ademas optima.\\
Los retailers que conforman el grupo $D_1$ son los retailers con numero: 2,4,5,9,10,16,18,21,22.
