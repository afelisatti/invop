\documentclass[11pt, a4paper, spanish]{article}

%%%%%%%%%% COMIENZO DEL PREAMBULO %%%%%%%%%%

%Info sobre este documento
\author{Ana Felisatti}
\title{Trabajo Pr'actico de Investigaci'on Operativa}

%\usepackage{infostyle}                                                  % provee un look & feel similar a un documento Word
\usepackage[top=2.5cm, bottom=2.5cm, left=2.5cm, right=2.5cm]{geometry}  % m\'argenes
\usepackage[ansinew]{inputenc}                                           % permite que los acentos del estilo \'a\'e\'i\'o\'u salgan joya
\usepackage[spanish, activeacute]{babel}                                 % idioma espa\~{n}ol, acentos f\'aciles y deletreo de palabras
\usepackage{indentfirst}                                                 % permite indentar un parrafo a mano
\usepackage{caratula}                                                    % incluye caratula est\'andar
\usepackage{graphicx}                                                    % permite insertar gr\'aficos
\usepackage{color}                                                       % permite el uso de colores en el documento
\usepackage[pdfcreator={TexLive!, LaTeX2e con TeXnicCenter},
			pdftitle={Investigaci'on Operativa},
			pdfsubject={Trabajo Practico},
			pdfstartview=FitH,            % Fits the width of the page to the window
			bookmarksnumbered,            % los bookmarks numerados se ven mejor...
			colorlinks,                   % links con bellos colores
			linkcolor=magenta]            % permite cambiar el color de los links
			{hyperref}                    % Permite jugar con algunas cosas que aparecer\'an en el PDF final
\usepackage{hyperref}
\usepackage{ulem}

%\selectlanguage{spanish}

\linespread{1.3}                    % interlineado equivalente al 1.5 l\'ineas de Word...
\pagestyle{myheadings}              %encabezado personalizable con \markboth{}{}
\markboth{}{Investigaci\'on Operativa - Felisatti, Raffo}
\headsep = 30pt                     % separaci\'on entre encabezado y comienzo del p\'arrafo

%\addtolength{\oddsidemargin}{-2cm}	% configuracion IDEAL!!!
%\addtolength{\textwidth}{4cm}
%\addtolength{\textheight}{2cm}

% macro 'todo' para To-Do's
\def\todo#1{\textcolor{red}{#1}}

% Macro 'borde' para un texto con borde
\newsavebox{\fmbox}
\newenvironment{borde}[1]
{\begin{lrbox}{\fmbox}\begin{minipage}{#1}}
{\end{minipage}\end{lrbox}\fbox{\usebox{\fmbox}}\\[10pt]}

% para que la gramatica quede linda
\newenvironment{xlist}[1]{%
	\begin{list}{}{%
		\settowidth{\labelwidth}{#1}
		\setlength{\labelsep}{0.5cm}
		\setlength{\leftmargin}{\labelwidth}
		\addtolength{\leftmargin}{\labelsep}
		\setlength{\rightmargin}{0pt}
		\setlength{\parsep}{0.5ex plus0.2ex minus0.1ex}
		\setlength{\itemsep}{0ex plus0.2ex}
		}
	}
{\end{list}}%%%%%%%%%% FIN DEL PREAMBULO %%%%%%%%%%

\begin{document}

\materia{Investigaci\'on Operativa}
\titulo{Trabajo Pr\'actico}
\subtitulo{}
\grupo{}

\integrante{Felisatti, Ana}{335/10}{anafelisatti@gmail.com}
\integrante{Raffo, Diego Martin}{335/10}{enanodr@gmail.com}


%\begin{center}
%	\includegraphics[scale=0.35]{diagramas/chuck-norris-thumbs-up.png}\\
%\end{center}


\maketitle

\thispagestyle{empty}

\newpage

% Conviene poner las secciones como diferentes archivos,
% sobre todo cuando se trabaja en equipo.
% Es m\'as f\'acil para sincronizar mediante control de versiones.
\tableofcontents

\newpage

\section{Ejercicio 12.1}
\subsection{Modelo}
\subsubsection{Variables}
Para cada tipo de aceite (VEG1, VEG2, OIL1, OIL2, OIL3) y cada mes (de enero a junio) dintinguimos 3 variables que denotan la cantidad comprada (Bought), refinada (Refined) y guardada (Saved). Por ejemplo, tenemos las variables $3\_VEG2\_Bought, 3\_VEG2\_Refined$ y $3\_VEG2\_Saved$ para el mes de marzo y el aceite VEG2.

En t'erminos generales, tenemos variables M\_A\_T donde M denota el mes (de 1 a 6), A donota el tipo de aceite y T el tipo de uso. 
\subsubsection{Restricciones}
Considerando las limitaciones de refinaci'on para aceites vegetales y no vegetales, para cada mes M tenemos:
\begin{itemize}
\item$M\_VEG1\_Refined + M\_VEG2\_Refined \leq 200$
\item$M\_OIL1\_Refined + M\_OIL2\_Refined + M\_OIL3\_Refined \leq 250$
\end{itemize}
Luego, por las limitaciones de espacio para guardarlos tenemos para cada mes M y cada tipo de aceite A:
\begin{itemize}
\item$M\_A\_Saved \leq 1000$
\end{itemize}
En t'erminos de dureza, tenemos para cada mes M, con $D_A$ el valor de la dureza del aceite A:
\begin{itemize}
\item$3 M\_Total\_Refined \leq  \sum_{A}^{} D_{A} M\_A\_Refined \leq 6 M\_Total\_Refined$
\end{itemize}
Pero considerando que el producto final ser'a la mezcla de las partes refinadas de cada aceite (es decir la misma sumatoria), separando las desigualdades nos queda:
\begin{itemize}
\item$0 \leq  \sum_{A}^{} (D_{A} - 3) M\_A\_Refined$
\item$\sum_{A}^{} (D_{A} - 6) M\_A\_Refined \leq 0$
\end{itemize}
Finalmente debemos relacionar los tres tipos de acciones posibles, esto es: la cantidad guardada el mes anterior y la comprada en el mes actual deben ser igual a lo refinado y guardado en el mes actual. Es decir, para cada mes M y cada tipo de aceite A:
\begin{itemize}
\item$M\_A\_Bought + (M-1)\_A\_Saved = M\_A\_Refined + M\_A\_Saved$
\end{itemize}
Considerando que las cantidades iniciales son 500 toneladas de cada aceite y que al finalizar tambi'en deben quedar 500 toneladas debemos agregar, para cada aceite A:
\begin{itemize}
\item$0\_A\_Saved = 500$
\item$6\_A\_Saved = 500$
\end{itemize}
\subsubsection{Objetivo}
Nuestro objetivo es maximizar la ganancia total, que es la sumatoria de cada ganancia mensual, donde tenemos que cada tonelada de producto final (la mezcla de los aceites refinados) es 150 libras, el costo de almacenamiento es 5 libras (para las toneladas guardadas) y que cada tipo de aceite tiene un cierto valor en el mercado seg'un el mes (para las toneladas compradas). Entonces, con $P_{A\_M}$ el precio del aceite A en el mes M tenemos que maximizar:
$$\sum_{M=1}^{6} \sum_{A}^{} 150 M\_A\_Refined - 5 M\_A\_Saved - P_{A\_M} M\_A\_Bought$$
\subsection{Soluci'on}
Obtuvimos el siguiente resultado: Status = Optimal, Objective = 107842.59259259258\\
Con la siguiente asignaci'on de variables (por simplicidad no listamos aquellas cuyo resultado fue 0):\\
\begin{Parallel}[v]{0.48\textwidth}{0.48\textwidth}
\ParallelLText{\noindent
$%1\_Oil1\_Bought = 0.0\\
%1\_Oil1\_Refined = 0.0\\
1\_Oil1\_Saved = 500.0\\
%1\_Oil2\_Bought = 0.0\\
1\_Oil2\_Refined = 250.0\\
1\_Oil2\_Saved = 250.0\\
%1\_Oil3\_Bought = 0.0\\
%1\_Oil3\_Refined = 0.0\\
1\_Oil3\_Saved = 500.0\\
%1\_Veg1\_Bought = 0.0\\
1\_Veg1\_Refined = 22.22222222222217\\
1\_Veg1\_Saved = 477.7777777777778\\
%1\_Veg2\_Bought = 0.0\\
1\_Veg2\_Refined = 177.77777777777783\\
1\_Veg2\_Saved = 322.2222222222222\\
%2\_Oil1\_Bought = 0.0\\
%2\_Oil1\_Refined = 0.0\\
2\_Oil1\_Saved = 500.0\\
2\_Oil2\_Bought = 750.0\\
2\_Oil2\_Refined = 250.0\\
2\_Oil2\_Saved = 750.0\\
%2\_Oil3\_Bought = 0.0\\
%2\_Oil3\_Refined = 0.0\\
2\_Oil3\_Saved = 500.0\\
%2\_Veg1\_Bought = 0.0\\
%2\_Veg1\_Refined = 0.0\\
2\_Veg1\_Saved = 477.7777777777778\\
%2\_Veg2\_Bought = 0.0\\
2\_Veg2\_Refined = 200.0\\
2\_Veg2\_Saved = 122.22222222222217\\
%3\_Oil1\_Bought = 0.0\\
%3\_Oil1\_Refined = 0.0\\
3\_Oil1\_Saved = 500.0\\
%3\_Oil2\_Bought = 0.0\\
3\_Oil2\_Refined = 250.0\\
3\_Oil2\_Saved = 500.0\\
%3\_Oil3\_Bought = 0.0\\
%3\_Oil3\_Refined = 0.0\\
3\_Oil3\_Saved = 500.0\\
%3\_Veg1\_Bought = 0.0\\
3\_Veg1\_Refined = 159.25925925925927\\
3\_Veg1\_Saved = 318.51851851851853\\
%3\_Veg2\_Bought = 0.0\\
3\_Veg2\_Refined = 40.74074074074073\\
3\_Veg2\_Saved = 81.48148148148144$
}
\ParallelRText{\noindent
$%4\_Oil1\_Bought = 0.0\\
%4\_Oil1\_Refined = 0.0\\
4\_Oil1\_Saved = 500.0\\
%4\_Oil2\_Bought = 0.0\\
4\_Oil2\_Refined = 250.0\\
4\_Oil2\_Saved = 250.0\\
%4\_Oil3\_Bought = 0.0\\
%4\_Oil3\_Refined = 0.0\\
4\_Oil3\_Saved = 500.0\\
%4\_Veg1\_Bought = 0.0\\
4\_Veg1\_Refined = 159.25925925925927\\
4\_Veg1\_Saved = 159.25925925925927\\
%4\_Veg2\_Bought = 0.0\\
4\_Veg2\_Refined = 40.74074074074073\\
4\_Veg2\_Saved = 40.740740740740705\\
%5\_Oil1\_Bought = 0.0\\
%5\_Oil1\_Refined = 0.0\\
5\_Oil1\_Saved = 500.0\\
%5\_Oil2\_Bought = 0.0\\
5\_Oil2\_Refined = 250.0\\
%5\_Oil2\_Saved = 0.0\\
%5\_Oil3\_Bought = 0.0\\
%5\_Oil3\_Refined = 0.0\\
5\_Oil3\_Saved = 500.0\\
%5\_Veg1\_Bought = 0.0\\
5\_Veg1\_Refined = 159.25925925925927\\
%5\_Veg1\_Saved = 0.0\\
%5\_Veg2\_Bought = 0.0\\
5\_Veg2\_Refined = 40.74074074074073\\
%5\_Veg2\_Saved = 0.0\\
%6\_Oil1\_Bought = 0.0\\
%6\_Oil1\_Refined = 0.0\\
6\_Oil1\_Saved = 500.0\\
6\_Oil2\_Bought = 750.0\\
6\_Oil2\_Refined = 250.0\\
6\_Oil2\_Saved = 500.0\\
%6\_Oil3\_Bought = 0.0\\
%6\_Oil3\_Refined = 0.0\\
6\_Oil3\_Saved = 500.0\\
6\_Veg1\_Bought = 659.2592592592592\\
6\_Veg1\_Refined = 159.25925925925927\\
6\_Veg1\_Saved = 500.0\\
6\_Veg2\_Bought = 540.7407407407408\\
6\_Veg2\_Refined = 40.74074074074073\\
6\_Veg2\_Saved = 500.0$
}
\end{Parallel}



\section{Ejercicio 12.2}
\subsection{Modelo}
Para esta extension del ejercicio 12.1 mantenemos todo lo mismo, agregando algunas variables y restricciones extra.
\subsubsection{Variables}
Incluimos ahora para cada mes M y cada tipo de aceite A una variable booleana $M\_A$ que resprenta el uso de A en el mes M.
\subsubsection{Restricciones}
Sabemos que de usarse un aceite se debe usar m'as de 20 toneladas y por el ejercicio anterior tenemos un l'imite superior para los aceites vegetales y no vegetales (200 y 250 respectivamente). Entonces para cada mes M y cada tipo de aceite A, con $S_{A}$ el l'imite superior de A seg'un vegetal o no vegetal, tenemos:
\begin{itemize}
\item$20 M\_A \leq M\_A\_Refined$
\item$M\_A\_Refined \leq S_{A} M\_A$
\end{itemize}
Por otro lado, para garantizar que solo se mezclen 3 tipos de aceites, para cada mes M y tipo de aceite A tenemos:
\begin{itemize}
\item$\sum_{A} M\_A \leq 3$
\end{itemize}
Finalmente, para garantizar que si se usa VEG1 o VEG2 tambien se use OIL3, para cada mes M tenemos:
\begin{itemize}
\item$M\_VEG1 \leq M\_OIL3$
\item$M\_VEG2 \leq M\_OIL3$
\end{itemize}
\subsubsection{Objetivo}
El objetivo sigue siendo el mismo que en el ejercicio anterior: maximizar la ganancia total.
\subsection{Soluci'on}
Obtuvimos el siguiente resultado: Status = Optimal, Objective = 100278.7037037`0371\\
Con la siguiente asignaci'on de variables (por simplicidad no listamos aquellas cuyo resultado fue 0): \\
\begin{Parallel}[v]{0.48\textwidth}{0.48\textwidth}
\ParallelLText{\noindent
$%1\_Oil1 = -0.0\\
%1\_Oil1\_Bought = 0.0\\
%1\_Oil1\_Refined = 0.0\\
1\_Oil1\_Saved = 500.0\\
%1\_Oil2 = -0.0\\
%1\_Oil2\_Bought = 0.0\\
%1\_Oil2\_Refined = 0.0\\
1\_Oil2\_Saved = 500.0\\
1\_Oil3 = 1.0\\
%1\_Oil3\_Bought = 0.0\\
1\_Oil3\_Refined = 250.0\\
1\_Oil3\_Saved = 250.0\\
1\_Veg1 = 1.0\\
%1\_Veg1\_Bought = 0.0\\
1\_Veg1\_Refined = 85.18518518518528\\
1\_Veg1\_Saved = 414.8148148148147\\
1\_Veg2 = 1.0\\
%1\_Veg2\_Bought = 0.0\\
1\_Veg2\_Refined = 114.81481481481472\\
1\_Veg2\_Saved = 385.1851851851853\\
%2\_Oil1 = -0.0\\
%2\_Oil1\_Bought = 0.0\\
%2\_Oil1\_Refined = 0.0\\
2\_Oil1\_Saved = 500.0\\
%2\_Oil2 = -0.0\\
2\_Oil2\_Bought = 190.0\\
%2\_Oil2\_Refined = 0.0\\
2\_Oil2\_Saved = 690.0\\
2\_Oil3 = 1.0\\
%2\_Oil3\_Bought = 0.0\\
2\_Oil3\_Refined = 250.0\\
%2\_Oil3\_Saved = 0.0\\
2\_Veg1 = 1.0\\
%2\_Veg1\_Bought = 0.0\\
2\_Veg1\_Refined = 85.1851851851852\\
2\_Veg1\_Saved = 329.6296296296295\\
2\_Veg2 = 1.0\\
%2\_Veg2\_Bought = 0.0\\
2\_Veg2\_Refined = 114.8148148148148\\
2\_Veg2\_Saved = 270.3703703703705\\
%3\_Oil1 = -0.0\\
%3\_Oil1\_Bought = 0.0\\
%3\_Oil1\_Refined = 0.0\\
3\_Oil1\_Saved = 500.0\\
3\_Oil2 = 1.0\\
%3\_Oil2\_Bought = 0.0\\
3\_Oil2\_Refined = 230.0\\
3\_Oil2\_Saved = 460.0\\
3\_Oil3 = 1.0\\
3\_Oil3\_Bought = 40.0\\
3\_Oil3\_Refined = 20.0\\
3\_Oil3\_Saved = 20.0\\
%3\_Veg1 = -0.0\\
%3\_Veg1\_Bought = 0.0\\
%3\_Veg1\_Refined = 0.0\\
3\_Veg1\_Saved = 329.6296296296295\\
3\_Veg2 = 1.0\\
%3\_Veg2\_Bought = 0.0\\
3\_Veg2\_Refined = 200.0\\
3\_Veg2\_Saved = 70.3703703703705$
}
\ParallelRText{\noindent
$%4\_Oil1 = -0.0\\
%4\_Oil1\_Bought = 0.0\\
%4\_Oil1\_Refined = 0.0\\
4\_Oil1\_Saved = 500.0\\
4\_Oil2 = 1.0\\
%4\_Oil2\_Bought = 0.0\\
4\_Oil2\_Refined = 230.0\\
4\_Oil2\_Saved = 230.0\\
4\_Oil3 = 1.0\\
%4\_Oil3\_Bought = 0.0\\
4\_Oil3\_Refined = 20.0\\
%4\_Oil3\_Saved = 0.0\\
4\_Veg1 = 1.0\\
%4\_Veg1\_Bought = 0.0\\
4\_Veg1\_Refined = 155.0\\
4\_Veg1\_Saved = 174.6296296296295\\
%4\_Veg2 = -0.0\\
%4\_Veg2\_Bought = 0.0\\
%4\_Veg2\_Refined = 0.0\\
4\_Veg2\_Saved = 70.3703703703705\\
%5\_Oil1 = -0.0\\
%5\_Oil1\_Bought = 0.0\\
%5\_Oil1\_Refined = 0.0\\
5\_Oil1\_Saved = 500.0\\
5\_Oil2 = 1.0\\
%5\_Oil2\_Bought = 0.0\\
5\_Oil2\_Refined = 230.0\\
%5\_Oil2\_Saved = 0.0\\
5\_Oil3 = 1.0\\
5\_Oil3\_Bought = 540.0\\
5\_Oil3\_Refined = 20.0\\
5\_Oil3\_Saved = 520.0\\
5\_Veg1 = 1.0\\
%5\_Veg1\_Bought = 0.0\\
5\_Veg1\_Refined = 155.0\\
5\_Veg1\_Saved = 19.629629629629505\\
%5\_Veg2 = -0.0\\
%5\_Veg2\_Bought = 0.0\\
%5\_Veg2\_Refined = 0.0\\
5\_Veg2\_Saved = 70.3703703703705\\
%6\_Oil1 = -0.0\\
%6\_Oil1\_Bought = 0.0\\
%6\_Oil1\_Refined = 0.0\\
6\_Oil1\_Saved = 500.0\\
6\_Oil2 = 1.0\\
6\_Oil2\_Bought = 730.0\\
6\_Oil2\_Refined = 230.0\\
6\_Oil2\_Saved = 500.0\\
6\_Oil3 = 1.0\\
%6\_Oil3\_Bought = 0.0\\
6\_Oil3\_Refined = 20.0\\
6\_Oil3\_Saved = 500.0\\
%6\_Veg1 = 0.0\\
6\_Veg1\_Bought = 480.3703703703705\\
%6\_Veg1\_Refined = 0.0\\
6\_Veg1\_Saved = 500.0\\
6\_Veg2 = 1.0\\
6\_Veg2\_Bought = 629.6296296296296\\
6\_Veg2\_Refined = 200.0\\
6\_Veg2\_Saved = 500.0$
}
\end{Parallel}



\section{Ejercicio 12.6}
\subsection{Modelo}
\subsubsection{Variables}
Utilizamos una serie de variables para analizar la distribuci'on de barriles de los distintos tipos de productos. Tenemos entonces: 
\begin{itemize}
\item C1 y C2, representando al crude 1 y 2 respectivamente
\item  LN, MN y HN, representando light, medium y heavy naphthas
\item LNP, MNP y HNP, representando light, medium y heavy naphtas utilizadas para premium petrol
\item LNRe, MNRe y HNRe,representando light, medium y heavy naphtas utilizadas para regular petrol
\item LNR, MNR y HNR, representando light, medium y heavy naphthas reformadas
\item RG, representando a la gasolina reformada
\item RGP y RGRe, representando a la gasolina reformada utilizada para premium petrol y regular petrol
\item LiO y HO, representando light y heavy oils
\item LiOJF y LiOC representando light oil utilizado para jet fuel y para cracked oil
\item HOJFy HOC representando heavy oil utilizado para jet fuel y para cracked oil
\item R, representando al residuum
\item RJF y RLO, representando el residuum utilizado para jet fuel y para lube oil
\item CF y CO, representando la cracked gasoline y el cracked oil
\item COJF, representando el cracked oil utilizado para jet fuel
\item CGP y CGRe, representando la cracked gasoline utilizada para premium y regular petrol
\item PP, RP, JF, FO y LO representando premium petrol, regular petrol, jet fuel, fuel oil y lube oil
\end{itemize}
\subsubsection{Restricciones}
Para empezar debemos considerar las restricciones del crudo:
\begin{itemize}
\item$C1 \leq 20000$
\item$C2 \leq 30000$
\item$C1 + C2 \leq 45000$
\end{itemize}
Luego la contribuci'on de estos a las naftas y aceites:
\begin{itemize}
\item$LN = 0.1C1 + 0.15C2$
\item$MN = 0.2C1 + 0.25C2$
\item$HN = 0.2C1 + 0.18C2$
\item$LiO = 0.12C1 + 0.08C2$
\item$HO = 0.2C1 + 0.19C2$
\item$R = 0.13C1 + 0.12C2$
\end{itemize}
Seguimos con la distrubuci'on de cada nafta en premium petrol, regular petrol y gasolina reformada, junto con las limitaciones de esta 'ultima y su divisi'on en petrol:
\begin{itemize}
\item$LN = LNP + LNRe + LNR$
\item$MN = MNP + MNRe + MNR$
\item$HN = HNP + HNRe + HNR$
\item$LNR + MNR+ HNR \leq 10000$
\item$RG = 0.6LNR + 0.52MNR+ 0.45HNR$
\item$RG = RGP + RGRe$
\end{itemize}
Ahora lo mismo para los aceites, considerando que para el fuel oil se destina una cantidad constante y la limitaci'on de barriles que pueden pasar por el proceso de cracking:
\begin{itemize}
\item$LiO = LiOJF + 0.55FO + LiOC$
\item$HO = HOJF + 0.16FO + HOC$
\item$LiOC + HOC \leq 8000$
\end{itemize}
Sobre el proceso de cracking tambi'en debemos considerar la relaci'on entre los productos:
\begin{itemize}
\item$CO = 0.68LiOC + 0.75HOC$
\item$CO = COJF + 0.22FO$
\item$CG = 0.28LiOC + 0.2HOC$
\item$CG = CGP + CGRe$
\end{itemize}
Por otro lado, sobre el residuum tenemos:
\begin{itemize}
\item$R = RJF + 0.05FO + RLO$
\end{itemize}
Ahora podemos dar las restricciones para los productos finales. Para los petrols tenemos su composici'on, las garant'ias sobre su octanaje y sobre la proporci'on entre la producci'on de premium y regular petrol:
\begin{itemize}
\item$PP = LNP + MNP + HNP + RGP + CGP$
\item$RP = LNRe + MNRe + HNRe + RGRe + CGRe$
\item$94PP \leq 90LNP + 80MNP + 70HNP + 115RGP + 105CGP$
\item$84RP \leq 90LNRe + 80MNRe + 70HNRe + 115RGRe + 105CGRe$
\item$PP = 0.4RP$
\end{itemize}
Para el jet fuel tenemos su composici'on y la garant'ia sobre el vapour:
\begin{itemize}
\item$JF = LiOJF + HOJF + COJF + RJF$
\item$JF \geq LiOJF + 0.6HOJF + 1.5COJF + 0.05RJF$
\end{itemize}
Finalmente, para el lube oil tenemos su composici'on y la limitaci'on de su producci'on:
\begin{itemize}
\item$LO = 0.5RLO$
\item$500 \leq LO \leq 1000$
\end{itemize}
\subsubsection{Objetivo}
Nuestro objetivo es maximizar la ganancia de la venta de los productos finales. Es decir:
\begin{itemize}
\item$700PP + 600RP + 400JF + 350FO + 150LO$
\end{itemize}
\subsection{Soluci'on}
Obtuvimos el siguiente resultado: \\
Status = Optimal, Objective = 2.11365134768933E7\\
Con la siguiente asignaci'on de variables:\\
Variables: \\
C1 = 15000.0\\
C2 = 30000.0\\
CG = 1936.0\\
CGP = 0.0\\
CGRe = 1936.0\\
CO = 5706.0\\
COJF = 5706.0\\
FO = 0.0\\
HN = 8400.0\\
HNP = 0.0\\
HNR = 5406.861843697072\\
HNRe = 2993.138156302928\\
HO = 8700.0\\
HOC = 3800.0\\
HOJF = 4900.0\\
JF = 15156.0\\
LN = 6000.0\\
LNP = 5726.934236631985\\
LNR = 0.0\\
LNRe = 273.0657633680148\\
LO = 500.0\\
LiO = 4200.0\\
LiOC = 4200.0\\
LiOJF = 0.0\\
MN = 10500.0\\
MNP = 0.0\\
MNR = 0.0\\
MNRe = 10500.0\\
PP = 6817.778853133316\\
R = 5550.0\\
RG = 2433.0878296636824\\
RGP = 1090.8446165013304\\
RGRe = 1342.243213162352\\
RJF = 4550.0\\
RLO = 1000.0\\
RP = 17044.44713283329\\
\section{Ejercicio 12.13}
\subsection{Modelo}
\subsubsection{Variables}
Las variables ${D1}_i$, $i = 1, .. ,23$ representan si el retailer $i$ pertenece al grupo ${D}_1$. Son variables enteras. \\
%${D1}_1$
%${D1}_2$
%${D1}_3$
%${D1}_4$
%${D1}_5$
%${D1}_6$
%${D1}_7$
%${D1}_8$
%${D1}_9$
%${D1}_{10}$
%${D1}_{11}$
%${D1}_{12}$
%${D1}_{13}$
%${D1}_{14}$
%${D1}_{15}$
%${D1}_{16}$
%${D1}_{17}$
%${D1}_{18}$
%${D1}_{19}$
%${D1}_{20}$
%${D1}_{21}$
%${D1}_{22}$
%${D1}_{23}$  \\
Variables slack para las restricciones. Las variables plusXXX y minusXXX son las variables slack que permiten que los valores controlados por el grupo $D_1$ pueda variar entre 35 y 45 \% del total a controlar en cada 'area. \\
Variables slack sobre el control de Delivery Points:
\begin{itemize}
\item $minusDelivery$ 
\item $plusDelivery$ 
\end{itemize}
Variables slack sobre el control de los Oil Markets en la region 1:
\begin{itemize}
\item $minusOil1$ 
\item $plusOil1$ 
\end{itemize}
Variables slack sobre el control de los Oil Markets en la region 2:
\begin{itemize}
\item $minusOil2$ 
\item $plusOil2$ 
\end{itemize}
Variables slack sobre el control de los Oil Markets en la region 3:
\begin{itemize}
\item $minusOil3$ 
\item $plusOil3$ 
\end{itemize}
Variables slack sobre el control del Spirit Market:
\begin{itemize}
\item $minusSpirit$ 
\item $plusSpirit$ 
\end{itemize}
\subsubsection{Restricciones}
A continuaci'on detallamos las restricciones impuestas en el modelo y una breve explicaci'on sobre cada una. \\
El retailer $i$-'esimo o pertenece o no pertenece a $D_1$. Solo puede tomar valores 0 o 1. 
\begin{itemize}
\item $0 \leq {D1}_i \leq 1 ,  i = 1,..,23 $ \\
\end{itemize}
%$0 \leq {D1}_1 \leq 1$ \\
%$0 \leq {D1}_2 \leq 1$ \\
%$0 \leq {D1}_3 \leq 1$ \\
%$0 \leq {D1}_4 \leq 1$ \\
%$0 \leq {D1}_5 \leq 1$ \\
%$0 \leq {D1}_6 \leq 1$ \\
%$0 \leq {D1}_7 \leq 1$ \\
%$0 \leq {D1}_8 \leq 1$ \\
%$0 \leq {D1}_9 \leq 1$ \\
%$0 \leq {D1}_{10} \leq 1$ \\
%$0 \leq {D1}_{11} \leq 1$ \\
%$0 \leq {D1}_{12} \leq 1$ \\
%$0 \leq {D1}_{13} \leq 1$ \\
%$0 \leq {D1}_{14} \leq 1$ \\
%$0 \leq {D1}_{15} \leq 1$ \\
%$0 \leq {D1}_{16} \leq 1$ \\
%$0 \leq {D1}_{17} \leq 1$ \\
%$0 \leq {D1}_{18} \leq 1$ \\
%$0 \leq {D1}_{19} \leq 1$ \\
%$0 \leq {D1}_{20} \leq 1$ \\
%$0 \leq {D1}_{21} \leq 1$ \\
%$0 \leq {D1}_{22} \leq 1$ \\
%$0 \leq {D1}_{23} \leq 1$ \\
El grupo $D_1$ tiene entre el 35 y el 45 \% del total del total de Delivery Points 
\begin{itemize}
\item 11 ${D1}_1$ + 47 ${D1}_2$ + 44 ${D1}_3$ + 25 ${D1}_4$ + 10 ${D1}_5$ + 26${D1}_6$ + 26 ${D1}_7$ + 54 ${D1}_8$ + 18 ${D1}_9$ + 51 ${D1}_{10}$ + 20 ${D1}_{11}$ + 105 ${D1}_{12}$ + 7 ${D1}_{13}$ + 16 ${D1}_{14}$ + 34 ${D1}_{15}$ + 100 ${D1}_{16}$ + 50 ${D1}_{17}$ + 21 ${D1}_{18}$ + 11 ${D1}_{19}$ + 19 ${D1}_{20}$ + 14 ${D1}_{21}$ + 10 ${D1}_{22}$ + 11 ${D1}_{23}$ + $plusDelivery$ - $minusDelivery = $ 292
\end{itemize}
El grupo $D_1$ suma entre el 35 y el 45 \% del total del total de Spirit Markets
\begin{itemize}
\item 34 ${D1}_1$ + 411 ${D1}_2$ + 82 ${D1}_3$ + 157 ${D1}_4$ + 5 ${D1}_5$ + 183${D1}_6$ + 14 ${D1}_7$ + 215 ${D1}_8$ + 102 ${D1}_9$ + 21 ${D1}_{10}$ + 54 ${D1}_{11}$ + 0 ${D1}_{12}$ + 6 ${D1}_{13}$ + 96 ${D1}_{14}$ + 118 ${D1}_{15}$ + 112 ${D1}_{16}$ + 535 ${D1}_{17}$ + 8 ${D1}_{18}$ + 53 ${D1}_{19}$ + 28 ${D1}_{20}$ + 69 ${D1}_{21}$ + 65 ${D1}_{22}$ + 27 ${D1}_{23}$ + $plusSpirit$ - $minusSpirit = $ 958
\end{itemize}
El grupo $D_1$ suma entre el 35 y el 45 \% del total del total de los Oil Markets en las regiones 1, 2 y 3
\begin{itemize}
\item 9 ${D1}_1$ + 13 ${D1}_2$ + 14 ${D1}_3$ + 17 ${D1}_4$ + 18 ${D1}_5$ + 19${D1}_6$ + 23 ${D1}_7$ + 21 ${D1}_8$ + $plusOil1$ - $minusOil1 = $ 53.6
\item 9 ${D1}_9$ + 11 ${D1}_{10}$ + 17 ${D1}_{11}$ + 18 ${D1}_{12}$ + 18 ${D1}_{13}$ + 17 ${D1}_{14}$ + 22 ${D1}_{15}$ + 24 ${D1}_{16}$ + 36 ${D1}_{17}$ + 43 ${D1}_{18}$ + $plusOil2$ - $minusOil2 = $ 86
\item 6 ${D1}_{19}$ + 15 ${D1}_{20}$ + 15 ${D1}_{21}$ + 25 ${D1}_{22}$ + 39 ${D1}_{23}$ + $plusOil3$ - $minusOil3 = $ 40
\end{itemize}
El grupo $D_1$ tiene entre el 35 y el 45 \% del grupo A, que por ser entero son exactamente 3 retailers
\begin{itemize}
\item ${D1}_1 + {D1}_2 + {D1}_3 + {D1}_5 + {D1}_6 + {D1}_{10} + {D1}_{15} + {D1}_{20} = 3$
\end{itemize}
El grupo $D_1$ tiene entre el 35 y el 45 \% del grupo B, que por tener que ser entero son exactamente 6 retailers
\begin{itemize}
\item ${D1}_4 + {D1}_7 + {D1}_8 + {D1}_9 + {D1}_{11} + {D1}_{12} + {D1}_{13} + {D1}_{14} + {D1}_{15} + {D1}_{16} + {D1}_{17} + {D1}_{18} + {D1}_{19} + {D1}_{21} + {D1}_{22} + {D1}_{23}   = 6$
\end{itemize}
Por 'ultimo, las variables slack toman valores entre 0 y el 5 \% del total de cada grupo.
\begin{itemize}
\item $0 \leq minusDelivery \leq$ 36.5
\item $0 \leq plusDelivery \leq$ 36.5
\item $0 \leq minusOil1 \leq$ 6.7
\item $0 \leq plusOil1 \leq$ 6.7
\item $0 \leq minusOil2 \leq$ 10.75
\item $0 \leq plusOil2 \leq$ 10.75
\item $0 \leq minusOil3 \leq$ 5.0
\item $0 \leq plusOil3 \leq$ 5.0
\item $0 \leq minusSpirit \leq$ 119.75
\item $0 \leq plusSpirit \leq$ 119.75
\end{itemize}
\subsubsection{Objetivo}
Lo primero que se ped'ia era verificar si exist'ia una soluci'on factible. De ser posible, se pedian dos posibles objetivos.\\
El primer objetivo: Minimizar la suma de los desvios porcentuales de las separaciones 40/60.\\
Minimizar: 
\begin{itemize}
\item $\frac{100}{292} minusDelivery$ + $\frac{100}{292} plusDelivery$ + $\frac{100}{53.6} minusOil1$ + $\frac{100}{53.6} plusOil1$ + $\frac{100}{86} minusOil2$ + $\frac{100}{86} plusOil2$ + $\frac{100}{40} minusOil3$ + $\frac{100}{40} plusOil3$ + $\frac{100}{958} minusSpirit$ + $\frac{100}{958} plusSpirit$
\end{itemize}
El segundo objetivo que se ped'ia era minimizar el m'aximo desv'io porcentual.
Para esto se agrego al modelo una variable no entera:
\begin{itemize}
\item $maxDesvio$
\end{itemize}
Tambi'en se agregaron las siguientes restricciones: 
\begin{itemize}
\item $maxDesvios \leq \frac{100}{292} minusDelivery$ + $\frac{100}{292} plusDelivery$
\item $maxDesvios \geq \frac{100}{53.6} minusOil1$ + $\frac{100}{53.6} plusOil1$
\item $maxDesvios \geq  \frac{100}{86} minusOil2$ + $\frac{100}{86} plusOil2$
\item $maxDesvios \geq \frac{100}{40} minusOil3$ + $\frac{100}{40} plusOil3$
\item $maxDesvios \geq \frac{100}{958} minusSpirit$ + $\frac{100}{958} plusSpirit$
\end{itemize}
Por 'ultimo, la funci'on objetivo se escrib'io de la siguiente forma:
Minimizar
 \begin{itemize}
 \item $maxDesvios$
\end{itemize}
\subsection{Soluci'on}
A continuaci'on el resultado devuelto por nuestra implementaci'on del modelo junto con el primer objetivo.\\
Status = Optimal, Objective = 2.7976179692883845 \\
Variables: \\
\begin{Parallel}[v]{0.48\textwidth}{0.48\textwidth}
\ParallelLText{\noindent
D1\_1 = 1.0 \\
D1\_2 = 1.0 \\
D1\_3 = 1.0 \\
D1\_4 = 1.0 \\
D1\_5 = 0.0 \\
D1\_6 = -0.0 \\
D1\_7 = 0.0 \\
D1\_8 = 0.0 \\
D1\_9 = -0.0 \\
D1\_10 = 0.0 \\
D1\_11 = 1.0 \\
D1\_12 = 0.0 \\
minusDelivery = -0.0 \\
minusOil1 = 0.0 \\
minusOil2 = -0.0 \\
minusOil3 = -0.0 \\
minusSpirit = 34.0
}
\ParallelRText{\noindent
D1\_13 = 0.0 \\
D1\_14 = -0.0 \\
D1\_15 = 0.0 \\
D1\_16 = 1.0 \\
D1\_17 = -0.0 \\
D1\_18 = 1.0 \\
D1\_19 = -0.0 \\
D1\_20 = 0.0 \\
D1\_21 = 1.0 \\
D1\_22 = 1.0 \\
D1\_23 = 0.0 \\
\\
plusDelivery = 0.0 \\
plusOil1 = 0.6000000000000014 \\
plusOil2 = 2.0 \\
plusOil3 = -0.0 \\
plusSpirit = -0.0
}
\end{Parallel}
Podemos ver que la soluci'on no solo es factible, sino adem'as 'optima.\\
De esta forma, los retailers que conforman el grupo $D_{1}$ son los retailers con numero: 1,2,3,4,11,16,18,21,22.\\
Al correr el modelo con el segundo objetivo se obtuvo el siguiente resultado: \\
Status = Optimal, Objective = 0.547945205479452 \\
Variables:
\begin{Parallel}[v]{0.48\textwidth}{0.48\textwidth}
\ParallelLText{\noindent
D1\_1 = 0.0 \\
D1\_2 = 1.0 \\
D1\_3 = -0.0 \\
D1\_4 = 1.0 \\
D1\_5 = 1.0 \\
D1\_6 = 0.0 \\
D1\_7 = -0.0 \\
D1\_8 = -0.0 \\
D1\_9 = 1.0 \\
D1\_10 = 1.0 \\
D1\_11 = 0.0 \\
D1\_12 = -0.0 \\
minusDelivery = 4.0 \\
minusOil1 = 0.7342465753424656 \\
minusOil2 = 1.0890410958904109 \\
minusOil3 = 0.0 \\
minusSpirit = 0.0
}
\ParallelRText{\noindent
D1\_13 = 0.0 \\
D1\_14 = -0.0 \\
D1\_15 = 0.0 \\
D1\_16 = 1.0 \\
D1\_17 = -0.0 \\
D1\_18 = 1.0 \\
D1\_19 = 0.0 \\
D1\_20 = -0.0 \\
D1\_21 = 1.0 \\
D1\_22 = 1.0 \\
D1\_23 = 0.0 \\
maxDesvio = 0.547945205479452 \\
plusDelivery = 0.0 \\
plusOil1 = 6.334246575342467 \\
plusOil2 = 0.08904109589041087 \\
plusOil3 = 0.0 \\
plusSpirit = 8.0
}
\end{Parallel}
Nuevamente podemos observar que la soluci'on no solo es factible, sino adem'as 'optima.\\
Los retailers que conforman el grupo $D_1$ son los retailers con numero: 2,4,5,9,10,16,18,21,22.

\section{Ejercicio 12.15}
\section{Ejercicio 12.16}
\subsection{Modelo}
En esta extensi'on del ejercicio anterior debemos considerar los nuevos generadores, la reserva de agua y como ambas cosas afectar'an al objetivo y las restricciones de demanda anteriores. Como en el caso anterior, el periodo anterior al primero ser'a el 'ultimo.
\subsubsection{Variables}
Agregamos las siguientes variables para cada periodo P:
\begin{itemize}
\item $Reservoir\_PeriodP$, representando la medida de la reserva en el periodo P
\item $Pumping\_PeriodP$, representando los metros que se modifica la reserva en el periodo P
\item $RunningHA\_PeriodP$ y $RunningHB\_PeriodP$, representando si los generadores hidroel'ectricos A y B estan operando en el periodo P (variables booleanas)
\item $StartHA\_PeriodP$ y $StartHB\_PeriodP$, representando si los generadores hidroel'ectricos A y B son encendidos en el periodo P (variables booleanas)
\end{itemize}
\subsubsection{Restricciones}
En principio establecemos las reglas sobre el nivel de la reserva. Para todo periodo P, con $L_{G}$ en nivel que baja la reserva por operaci'on de la generadora $G$, tenemos:
\begin{itemize}
\item $Reservoir\_Period1 = 16$
\item $15 \leq Reservoir\_PeriodP \leq 20$
\item{
\begin{equation}
\begin{aligned}
Reservoir\_PeriodP = & Reservoir\_Period(P-1) \\
				& + Pumping\_PeriodP \\
				& - \sum_{G \in \{HA,HB\}} L_{G} RunningG\_PeriodP
\end{aligned}
\end{equation}
}
\end{itemize}
Por otro lado, la relaci'on entre las variables de operaci'on y encendido esta dada por, considerando a G el tipo de generador, para cada periodo P:
\begin{itemize}
\item $StartG\_PeriodP \geq RunningG\_PeriodP - RunningG\_Period(P-1)$
\end{itemize}
Finalmente, sobre la demanda y la demanda extra, debemos agregar al lado derecho de ~\ref{eq:demand} y ~\ref{eq:extrademand} los siguientes t'erminos, respectivamente (con $H_{P}$ la cantidad de horas en el periodo P):
\begin{itemize}
\item $900 RunningHA\_PeriodP + 1400 RunningHB\_PeriodP - 3000/H_{P} Pumping\_PeriodP$
\item $900 RunningHA\_PeriodP + 1400 RunningHB\_PeriodP$
\end{itemize}
En un principio creimos que tambi'en debiamos agregar a la demanda extra los MW dedicados a rellenar la reserva (calculados como $3000/H_{P} Pumping\_PeriodP$) pero finalmente nos dimos cuenta de que al estar siendo producidos por las generadoras anteriores, ya estan considerados en la formulaci'on de demanda extra. 
\subsubsection{Objetivo}
Para el objetivo debemos agregar a ~\ref{eq:objective} el costo asociado a las generadoras hidroel'ecticas. Considerando nuevamente $H_{P}$ como la cantidad de horas en el periodo P, tenemos:
\begin{equation} \label{eq:extendedobjective}
\begin{aligned}
& 90 H_{P} RunningHA\_PeriodP + 1500 StartHA\_PeriodP +\\
& 150 H_{P} RunningHB\_PeriodP + 1200 StartHB\_PeriodP \\
\end{aligned}
\end{equation}
\subsection{Soluci'on}
Obtuvimos el siguiente resultado:\\
Status = Optimall, Objective = 986880.0\\
Con la siguiente asignaci'on de variables (por simplicidad no mostramos aquellas cuyo resultado fue 0).\\
Variables: \\
Amount\_Type1\_Period1 = 12.0\\
Amount\_Type1\_Period2 = 12.0\\
Amount\_Type1\_Period3 = 12.0\\
Amount\_Type1\_Period4 = 12.0\\
Amount\_Type1\_Period5 = 12.0\\
Amount\_Type2\_Period1 = 3.0\\
Amount\_Type2\_Period2 = 9.0\\
Amount\_Type2\_Period3 = 9.0\\
Amount\_Type2\_Period4 = 10.0\\
Amount\_Type2\_Period5 = 9.0\\
%Amount\_Type3\_Period1 = 0.0\\
%Amount\_Type3\_Period2 = 0.0\\
%Amount\_Type3\_Period3 = 0.0\\
Amount\_Type3\_Period4 = 1.0\\
%Amount\_Type3\_Period5 = 0.0\\
Pumping\_Period1 = 0.9000000000000021\\
%Pumping\_Period2 = 0.0\\
Pumping\_Period3 = 1.9000000000000021\\
%Pumping\_Period4 = 0.0\\
Pumping\_Period5 = 1.4299999999999962\\
Reservoir\_Period1 = 16.0\\
Reservoir\_Period2 = 16.0\\
Reservoir\_Period3 = 17.900000000000002\\
Reservoir\_Period4 = 16.490000000000002\\
Reservoir\_Period5 = 15.099999999999998\\
%RunHA\_Period1 = -0.0\\
%RunHA\_Period2 = -0.0\\
%RunHA\_Period3 = 0.0\\
%RunHA\_Period4 = 0.0\\
%RunHA\_Period5 = -0.0\\
%RunHB\_Period1 = 0.0\\
%RunHB\_Period2 = -0.0\\
%RunHB\_Period3 = 0.0\\
RunHB\_Period4 = 1.0\\
RunHB\_Period5 = 1.0\\
%StartHA\_Period1 = -0.0\\
%StartHA\_Period2 = -0.0\\
%StartHA\_Period3 = -0.0\\
%StartHA\_Period4 = -0.0\\
%StartHA\_Period5 = -0.0\\
%StartHB\_Period1 = -0.0\\
%StartHB\_Period2 = -0.0\\
%StartHB\_Period3 = -0.0\\
StartHB\_Period4 = 1.0\\
%StartHB\_Period5 = -0.0\\
%Started\_Type1\_Period1 = 0.0\\
%Started\_Type1\_Period2 = 0.0\\
%Started\_Type1\_Period3 = 0.0\\
%Started\_Type1\_Period4 = 0.0\\
%Started\_Type1\_Period5 = 0.0\\
%Started\_Type2\_Period1 = 0.0\\
Started\_Type2\_Period2 = 6.0\\
%Started\_Type2\_Period3 = 0.0\\
Started\_Type2\_Period4 = 1.0\\
%Started\_Type2\_Period5 = 0.0\\
%Started\_Type3\_Period1 = 0.0\\
%Started\_Type3\_Period2 = 0.0\\
%Started\_Type3\_Period3 = 0.0\\
Started\_Type3\_Period4 = 1.0\\
%Started\_Type3\_Period5 = 0.0\\
Watts\_Type1\_Period1 = 10200.0\\
Watts\_Type1\_Period2 = 14250.0\\
Watts\_Type1\_Period3 = 10200.0\\
Watts\_Type1\_Period4 = 19600.0\\
Watts\_Type1\_Period5 = 10564.999999999998\\
Watts\_Type2\_Period1 = 5250.0\\
Watts\_Type2\_Period2 = 15750.0\\
Watts\_Type2\_Period3 = 15750.0\\
Watts\_Type2\_Period4 = 17500.0\\
Watts\_Type2\_Period5 = 15750.0\\
%Watts\_Type3\_Period1 = 0.0\\
%Watts\_Type3\_Period2 = 0.0\\
%Watts\_Type3\_Period3 = 0.0\\
Watts\_Type3\_Period4 = 1500.0\\
%Watts\_Type3\_Period5 = 0.0\\
\section{Ejercicio 12.23}
\section{Ejercicios Pr'actica 8}
\subsection{Ejercicio 20.a}
Encontrar una desigualdad v'alida para $X$ que corte el punto $x^{*}$ dado, donde
\begin{center}
$X = \{x \in B^{5} / 9x_1 + 8x_2 + 6x_3 + 6x_4 + 5x_5 \leq 14\}$ \\
$x^{*} = (0, 5/8, 3/4, 3/4, 0)$
\end{center}
Utilizando el m'etodo de Chv'atal-Gomory podemos multiplicar la desigualdad por $1/8$ y luego restar la parte entera a cada coeficiente:
\begin{center}
\begin{equation}
9/8 x_1 + 8/8 x_2 + 6/8 x_3 + 6/8 x_4 + 5/8 x_5 \leq 14/8
\end{equation}
\begin{equation}
9/8 x_1 + x_2 + 3/4 x_3 + 3/4 x_4 + 5/8 x_5 \leq 7/4
\end{equation}
\begin{equation} \label{eq:solve}
1/8 x_1 + 3/4 x_3 + 3/4 x_4 + 5/8 x_5 \leq 3/4
\end{equation}
\end{center}
Si ahora evaluamos la desigualdad ~\ref{eq:solve} en $x^{*}$ tenemos:
\begin{center}
$1/8\times 0 + 3/4\times 3/4 + 3/4\times 3/4 + 5/8\times 0 = 9/8 > 3/4$
\end{center}
Entonces ~\ref{eq:solve} es una desigualdad v'alida y corta a $x^{*}$.

\subsection{Ejercicio 22}
Dado un grafo $G = (V, E)$ con $n = |V|$ considerar el conjunto de vectores soluciones de $X$ del
problema de conjunto independiente $X = \{ x\in B^{n}/ x_{i}+x_{j}\leq 1$ para toda arista $e =(i,j) \}$. Mostrar que la desigualdad de clique $\sum_{j\in C} x_{j} \leq 1$ para $C$ clique maximal de $G$, es v'alida. 
 


% BEGIN Ejemplos de uso

	%\section{Una secci\'on}
	%\label{sec:unaSeccion}
	%Hola! Soy una Secci\'on
	%	\subsection{Una subsecci\'on}
	%		Y yo soy una subsecci\'on!!!
	%		\subsubsection{Una subsubsecci\'on}
	%			Y yo soy una sub-subsecci\'on!!!
	%			\paragraph{Un p\'arrafo\\}
	%				Y yo soy un p\'arrafo, porque no hay mas sub-sub-sub-subsecciones!!!

	%\section{Otra secci\'on}
	%	Como pudimos ver en la secci\'on \ref{sec:unaSeccion}, esto es una demo de una referencia a una secci\'on.
	
	%	Tambi\'en podemos hacer referencia a la p\'agina de la secci\'on:\\[10pt]
	
		% Ejemplo de uso de un borde (falta pulir para que no tire un warning!)
	%	\begin{borde}{0.98\textwidth}
	%		En la p\'agina \pageref{sec:unaSeccion}, hay una secci\'on pilla...
	%	\end{borde}

% END Ejemplos de uso


\end{document}
