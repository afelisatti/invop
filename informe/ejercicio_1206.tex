\section{Ejercicio 12.6}
\subsection{Modelo}
\subsubsection{Variables}
Utilizamos una serie de variables para analizar la distribuci'on de barriles de los distintos tipos de productos. Tenemos entonces: 
\begin{itemize}
\item C1 y C2, representando al crude 1 y 2 respectivamente
\item  LN, MN y HN, representando light, medium y heavy naphthas
\item LNP, MNP y HNP, representando light, medium y heavy naphtas utilizadas para premium petrol
\item LNRe, MNRe y HNRe,representando light, medium y heavy naphtas utilizadas para regular petrol
\item LNR, MNR y HNR, representando light, medium y heavy naphthas reformadas
\item RG, representando a la gasolina reformada
\item RGP y RGRe, representando a la gasolina reformada utilizada para premium petrol y regular petrol
\item LiO y HO, representando light y heavy oils
\item LiOJF y LiOC representando light oil utilizado para jet fuel y para cracked oil
\item HOJFy HOC representando heavy oil utilizado para jet fuel y para cracked oil
\item R, representando al residuum
\item RJF y RLO, representando el residuum utilizado para jet fuel y para lube oil
\item CF y CO, representando la cracked gasoline y el cracked oil
\item COJF, representando el cracked oil utilizado para jet fuel
\item CGP y CGRe, representando la cracked gasoline utilizada para premium y regular petrol
\item PP, RP, JF, FO y LO representando premium petrol, regular petrol, jet fuel, fuel oil y lube oil
\end{itemize}
\subsubsection{Restricciones}
Para empezar debemos considerar las restricciones del crudo:
\begin{itemize}
\item$C1 \leq 20000$
\item$C2 \leq 30000$
\item$C1 + C2 \leq 45000$
\end{itemize}
Luego la contribuci'on de estos a las naftas y aceites:
\begin{itemize}
\item$LN = 0.1C1 + 0.15C2$
\item$MN = 0.2C1 + 0.25C2$
\item$HN = 0.2C1 + 0.18C2$
\item$LiO = 0.12C1 + 0.08C2$
\item$HO = 0.2C1 + 0.19C2$
\item$R = 0.13C1 + 0.12C2$
\end{itemize}
Seguimos con la distrubuci'on de cada nafta en premium petrol, regular petrol y gasolina reformada, junto con las limitaciones de esta 'ultima y su divisi'on en petrol:
\begin{itemize}
\item$LN = LNP + LNRe + LNR$
\item$MN = MNP + MNRe + MNR$
\item$HN = HNP + HNRe + HNR$
\item$LNR + MNR+ HNR \leq 10000$
\item$RG = 0.6LNR + 0.52MNR+ 0.45HNR$
\item$RG = RGP + RGRe$
\end{itemize}
Ahora lo mismo para los aceites, considerando que para el fuel oil se destina una cantidad constante y la limitaci'on de barriles que pueden pasar por el proceso de cracking:
\begin{itemize}
\item$LiO = LiOJF + 0.55FO + LiOC$
\item$HO = HOJF + 0.16FO + HOC$
\item$LiOC + HOC \leq 8000$
\end{itemize}
Sobre el proceso de cracking tambi'en debemos considerar la relaci'on entre los productos:
\begin{itemize}
\item$CO = 0.68LiOC + 0.75HOC$
\item$CO = COJF + 0.22FO$
\item$CG = 0.28LiOC + 0.2HOC$
\item$CG = CGP + CGRe$
\end{itemize}
Por otro lado, sobre el residuum tenemos:
\begin{itemize}
\item$R = RJF + 0.05FO + RLO$
\end{itemize}
Ahora podemos dar las restricciones para los productos finales. Para los petrols tenemos su composici'on, las garant'ias sobre su octanaje y sobre la proporci'on entre la producci'on de premium y regular petrol:
\begin{itemize}
\item$PP = LNP + MNP + HNP + RGP + CGP$
\item$RP = LNRe + MNRe + HNRe + RGRe + CGRe$
\item$94PP \leq 90LNP + 80MNP + 70HNP + 115RGP + 105CGP$
\item$84RP \leq 90LNRe + 80MNRe + 70HNRe + 115RGRe + 105CGRe$
\item$PP = 0.4RP$
\end{itemize}
Para el jet fuel tenemos su composici'on y la garant'ia sobre el vapour:
\begin{itemize}
\item$JF = LiOJF + HOJF + COJF + RJF$
\item$JF \geq LiOJF + 0.6HOJF + 1.5COJF + 0.05RJF$
\end{itemize}
Finalmente, para el lube oil tenemos su composici'on y la limitaci'on de su producci'on:
\begin{itemize}
\item$LO = 0.5RLO$
\item$500 \leq LO \leq 1000$
\end{itemize}
\subsubsection{Objetivo}
Nuestro objetivo es maximizar la ganancia de la venta de los productos finales. Es decir:
\begin{itemize}
\item$700PP + 600RP + 400JF + 350FO + 150LO$
\end{itemize}
\subsection{Soluci'on}
Obtuvimos el siguiente resultado: \\
Status = Optimal, Objective = 2.11365134768933E7\\
Con la siguiente asignaci'on de variables:\\
Variables: \\
C1 = 15000.0\\
C2 = 30000.0\\
CG = 1936.0\\
CGP = 0.0\\
CGRe = 1936.0\\
CO = 5706.0\\
COJF = 5706.0\\
FO = 0.0\\
HN = 8400.0\\
HNP = 0.0\\
HNR = 5406.861843697072\\
HNRe = 2993.138156302928\\
HO = 8700.0\\
HOC = 3800.0\\
HOJF = 4900.0\\
JF = 15156.0\\
LN = 6000.0\\
LNP = 5726.934236631985\\
LNR = 0.0\\
LNRe = 273.0657633680148\\
LO = 500.0\\
LiO = 4200.0\\
LiOC = 4200.0\\
LiOJF = 0.0\\
MN = 10500.0\\
MNP = 0.0\\
MNR = 0.0\\
MNRe = 10500.0\\
PP = 6817.778853133316\\
R = 5550.0\\
RG = 2433.0878296636824\\
RGP = 1090.8446165013304\\
RGRe = 1342.243213162352\\
RJF = 4550.0\\
RLO = 1000.0\\
RP = 17044.44713283329\\